\chapter{INTRODUÇÃO}

Este trabalho viza apresentar as experiencias vividas durante o periodo como estagiário do o LEMAF (Laboratório de Projetos e Estudos
em Manejo Florestal), laboratório situado dentro da Universidade Federal de Lavras(UFLA) onde são desenvolvidas soluções tecnológicas relacionadas a manejo florestal.
O laboratório foi inalgurado em 2002 e desde lá, vem desenvolvimento diversos projetos para diversos orgãos e empresas.

Os projetos do LEMAF são no geral ligados ao meio ambiente, como monitoramento de areas desmatadas e uso indevido de recursos hidricos, porem seu arcenal de projetos inclui diversos temas, como portais para povos indigenas, monitoramento de numero de animais atropelados por região e ate mesmo gerenciadores de conteudos.   
O estágio proporcionado pelo mesmo tinha como responsabilidades o seguimento de metodologias scrum, organização e trabalho em equipe e principalmente o desenvovimento de projetos web.

Por conta de uma grande rotação de projetos e times, foi nescessario o estudo de diversos frameworks e tecnologias, onde os principais relacionados com frontend foram Angular, Vuejs e ReactJs, enquanto com backend foram SpringBoot, DotNet Framework e PlayFramework.
Para que o desenvovimento dos projetos fluissem efetivamente, eram utilizadas diversas metodologias ageis, como principais Scrum e Kambam.