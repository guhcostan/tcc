\chapter{INTRODUÇÃO}
\label{cap:introducao}

Este trabalho visa apresentar as experiências vividas durante o período como estagiário do LEMAF (Laboratório de Projetos e Estudos
em Manejo Florestal), laboratório situado dentro da Universidade Federal de Lavras (UFLA) onde são desenvolvidas soluções tecnológicas relacionadas a manejo florestal.
O laboratório foi fundado em 2004 e, desde lá, desenvolve diversos projetos para órgãos governamentais e empresas privadas, contando em 2018 com mais de 160 funcionários.

Os projetos do LEMAF geralmente possuem contextos relacionados à preservação ambiental, como monitoramento de áreas desmatadas e uso indevido de recursos hídricos. Porém sua carteira de projetos inclui diversos temas, como portais para povos indígenas, monitoramento de número de animais atropelados em determinada região e até mesmo gerenciadores de conteúdos.   
O estágio proporcionado pelo mesmo tinha como responsabilidades o seguimento de metodologias ágeis, organização, trabalho em equipe e, principalmente, o desenvolvimento de projetos web.

Por conta de uma grande rotação de projetos e times, foi necessário o estudo de diversos \textit{frameworks} e tecnologias, onde os principais relacionados com \textit{frontend} foram Angular, Vuejs e ReactJS, enquanto com \textit{backend} foram SpringBoot, DotNet \textit{Framework} e Play\textit{Framework}.
Para que o desenvolvimento dos projetos fluíssem efetivamente, eram utilizadas diversas metodologias ágeis, como principais \textit{Scrum} e \textit{Kanban}.

Com toda essa carga de conhecimento e responsabilidades, ser estagiário no LEMAF se tornou uma experiência incrível. Estava sempre evoluindo, conhecendo novas tecnologias do mercado e tendo oportunidade de obter conhecimento de pessoas extremamente experientes no ramo.

Os demais capítulos deste relatório estão organizados da seguinte forma.  No Capítulo 2 são abordadas as ferramentas e processos utilizados durante o estágio. No Capítulo 3 são apresentadas as atividades desenvolvidas durante o estágio. No Capítulo 4 são apresentadas conclusões.