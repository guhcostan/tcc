\chapter{INTRODUÇÃO}
\label{cap:introducao}

Este trabalho visa apresentar as experiências vividas durante o período como estagiário do LEMAF (Laboratório de Projetos e Estudos
em Manejo Florestal), laboratório situado dentro da Universidade Federal de Lavras (UFLA) onde são desenvolvidas soluções tecnológicas relacionadas a manejo florestal.
O laboratório foi inaugurado em 2002 e, desde lá, desenvolve diversos projetos para órgãos governamentais e empresas privadas, contando em 2018 com mais de 100 funcionários.

Os projetos do LEMAF geralmente possuem contextos relacionados à preservação ambiental, como monitoramento de áreas desmatadas e uso indevido de recursos hídricos. Porém seu arsenal de projetos inclui diversos temas, como portais para povos indígenas, monitoramento de número de animais atropelados em determinada região e até mesmo gerenciadores de conteúdos.   
O estágio proporcionado pelo mesmo tinha como responsabilidades o seguimento de metodologias ágeis, organização, trabalho em equipe e, principalmente, o desenvolvimento de projetos web.

Por conta de uma grande rotação de projetos e times, foi necessário o estudo de diversos frameworks e tecnologias, onde os principais relacionados com frontend foram Angular, Vuejs e ReactJs, enquanto com backend foram SpringBoot, DotNet Framework e PlayFramework.
Para que o desenvolvimento dos projetos fluíssem efetivamente, eram utilizadas diversas metodologias ágeis, como principais Scrum e Kanban.

Com toda essa carga de conhecimento e responsabilidades, ser estagiário no LEMAF se tornou uma experiencia incrível, uma vez que não era tratado como tal. Estava sempre evoluindo, conhecendo novas tecnologias do mercado e tendo oportunidade de obter conhecimento de pessoas extremamente experientes no ramo.