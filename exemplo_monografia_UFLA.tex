\documentclass{uflamon}          % classe base para a monografia

%==============================================================================
% Utilizacao de pacotes
\usepackage[T1]{fontenc}         % usa fontes postscript com acentos
\usepackage[brazil]{babel}       % hifenização e títulos em português do Brasil
\usepackage[utf8]{inputenc}     % permite edição direta com acentos
\usepackage{amsmath}             % pacote da AMS para Matemática Avançada
\usepackage{amssymb}             % símbolos extras da AMS
\usepackage{latexsym}            % símbolos extras do LaTeX
\usepackage{graphicx}            % para inserção de gráficos
\usepackage{listings}            % para inserção de código
\usepackage{fancyvrb}            % para inserção de saídas de comandos
%\usepackage{enumerate}           % para personalizar lista enumeradas 
											%(incluso na classe)
\usepackage{longtable}           % para tambelas muito grandes NOVO!!!!

\usepackage{colortbl} % cores em tabelas
\newcolumntype{Z}{|>{\columncolor[gray]{0.9}}l|} %cor cinza em células
%\usepackage{array} % já incluso na classe
\newcolumntype{L}[1]{>{\raggedright\let\newline\\\arraybackslash\hspace{0pt}}m{#1}}
\newcolumntype{C}[1]{>{\centering\let\newline\\\arraybackslash\hspace{0pt}}m{#1}}
\newcolumntype{R}[1]{>{\raggedleft\let\newline\\\arraybackslash\hspace{0pt}}m{#1}}
\usepackage{multirow} % para juntar duas linhas em uma só

\usepackage{multicol} % para uso de várias colunas

% cores para os links cruzados
\usepackage{color}
\definecolor{rltred}{rgb}{0.2,0,0}
\definecolor{rltgreen}{rgb}{0,0.2,0}
\definecolor{rltblue}{rgb}{0,0,0.2}

\usepackage[colorlinks=true,
            urlcolor=rltblue,       % \href{...}{...} external (URL)
            filecolor=rltgreen,     % \href{...} local file
            linkcolor=rltred,       % \ref{...} and \pageref{...}
            citecolor=rltgreen,
            pdftitle={Relatório Estágio LEMAF - Gustavo Costa Neves},
          pdfauthor={Gustavo Costa Neves},
          pdfsubject={Este texto tem por objetivo servir de exemplo da classe Uflamon.},
          pdfkeywords={Comunicação Científica. 2. Pesquisa . 3. Pesquisa Científica. 
 					 4. Redação. 5. Monografia.}%
]{hyperref} % para referência cruzadas
%\usepackage{hyperref}            % para referência cruzadas
\usepackage{subfigure}           % figuras dentro de figuras
\usepackage{caption}            % remodelando o formato dos títulos de 
                                 % tabelas e figuras

% configuração padrão do listings   
\lstset{
   language=Java,
   extendedchars=true,
   tabsize=3,
   basicstyle=\footnotesize\ttfamily,
   stringstyle=\em,
   showstringspaces=false 
}

% para referências de acordo com a ABNT
% precisa instalar o abntex2 antes!!!
% http://abntex.codigolivre.org.br/
% comente se pretende usar outro padrão

%abnt-emphasize=bf coloca o título das bibliografias em negrito
%abnt-thesis-year=both
\usepackage[alf,abnt-etal-cite=3,abnt-etal-list=3,abnt-url-package=url,abnt-emphasize=bf]{abntex2cite}

% evite usar o hyperref com abntex, pode dar caca em urls... no linha anterior, informo
% para incluir urls usando o pacote url e não o hyperref
%
% caso queira o hyperref com abntex, comente a linha anterior e descomente a seguinte
%\usepackage[alf,abnt-etal-cite=3,abnt-etal-list=0,abnt-etal-text=emph]{abntex2cite}
%
% caso vc ainda use a versão anterior da abntex, comente a linha incluindo o abntex2cite
% e descomente a próxima linha 
%\usepackage[alf,abnt-etal-cite=3,abnt-etal-list=0,abnt-etal-text=emph]{abntcite}


% redefinindo formatação de títulos de tabelas e figuras


%==============================================================================
% para os fãs do Word, descomente as linhas abaixo
%\sloppy %mais espaço entre as linhas
%\usepackage{identfirst} %identando-se a primeira linha de cada seção
%\noindentfirst % Tire o comentário para manter o padrão do LaTeX.

%==============================================================================
% definido comandos na monografia - não é necessário na sua monografia 
% apenas para exemplificar a definição de novos comandos
\newcommand{\defs}[1]{\textsl{#1}}


% Especificando hifenizações que por ventura LaTeX não saiba fazer
% Por padrão 99,9% dos termos em português devem ser hifenizados corretamente.
\hyphenation{hardware software Li-nux am-bien-te diag-nos-ti-car coor-de-na-ção 
FAE-PE Recovery TelEduc Williams UFLA}

%==============================================================================
% Dados da monografia, capa: autor, titulo, banca, etc... - SUBSTITUA DE ACORDO
%==============================================================================
\author{Gustavo Costa Neves}
\title{ESTÁGIO EM DESENVOLVIMENTO DE SISTEMAS WEB NO LEMAF/UFLA}
\date{2019}
\tipo{Relatório de estágio apresentado na
Universidade Federal de Lavras – UFLA
como requisito para obtenção do título de
graduação no curso de Ciência da computação, sob orientação da Professora
Marluce Rodrigues Pereira.
}
% use \orientador ou \orientadora quando for o caso
\orientadora{Prof. Marluce Rodrigues Pereira}
%\orientadora{}
% use \coorientador ou \coorientadora quando for o caso
%\coorientadora{Prof. DSc. Maria Orientadora } % comente se não tiver coorientador
%\coorientador{}
\local{Lavras -- MG}
\bancaum{Prof. MSc. Antônio Banca Um}{UFM}
\bancadois{Prof. DSc. João Banca Dois}{FCO} % comente se sua banca tiver só um professor
\bancatres{Profa. Esp. Eliza Banca Três}{BELMIS}
\bancaquatro{Prof. Esp. Carlos Banca Quatro}{IBGPLUS}
\defesa{30 de Fevereiro de 2019}
%==============================================================================


%##################################################

%\antesfichacat{\noindent Para citar este documento: \\UNIVERSIDADE FEDERAL DE LAVRAS. Biblioteca Universitária. \textbf{Manual de normalização e estrutura de trabalhos acadêmicos: TCC, monografias, dissertações e teses}. 2. ed. rev., atual. e ampl. Lavras, 2015. Disponível em: \url{http://www.biblioteca.ufla.br/wordpress/wpcontent/uploads/bdtd/manual_normalizacao_UFLA.pdf}. Acesso em: data de acesso.}

%\depoisfichacat{\noindent A reprodução e a divulgação total ou parcial deste trabalho são autorizadas, por qualquer meio convencional ou eletrônico, para fins de estudo e pesquisa, desde que citada a fonte.\\
%\newline
%{\small Este documento possui páginas em branco para facilitar a impressão frente-e-verso.}}

%##################################################

%##################################################

% para os exemplos do manual
%\newenvironment{exemplomanual}{
%\vspace{0.5cm}
%\noindent\begin{minipage}{\textwidth}
%\noindent\rule{\textwidth}{0.5pt}
%\vspace{-1cm}
%\begin{flushleft}
%}{
%\end{flushleft}
%\vspace{-0.6cm}
%\noindent\rule{\textwidth}{0.5pt}
%\vspace{0.3cm}
%\end{minipage}
%}

%\newenvironment{exemplomanuallista}{
%\vspace{0.3cm}
%\noindent\begin{minipage}{\textwidth - 0.5cm}
%\noindent\rule{\textwidth}{0.5pt}
%\vspace{-1cm}
%\begin{flushleft}
%}{
%\end{flushleft}
%\vspace{-0.6cm}
%\noindent\rule{\textwidth}{0.5pt}
%\vspace{0.3cm}
%\end{minipage}
%}

% por conta de alguns exemplos
%\usepackage{setspace}

%##################################################

% se vc já defendeu e tem o arquivo escaneado da folha de rosto, 
% descomente e altere o nome do arquivo
%\folhaAprovacaoAssinada{folharosto}

% Aqui começa o documento propriamente dito
\begin{document}

\maketitle

\dedic{Dedico esse trabalho de conclusão de curso aos meus pais que sempre me apoiaram em toda essa tragetória}     % Dedicatórias\\

\thanks{Agradeço a todos que estiveram contribuindo comigo durante esse periodo de ufla.
\\
\\
Agradecimentos especiais para: 
\\

\quad Bruna Gomes.

\quad Danilo Guarizzo.

\quad Ivan Carvalho.

\quad Bruno Custódio.

\quad Highlander Silva.

\quad Matheus Flausino.

\quad Amanda Lima.

\quad Todos funcionarios do LEMAF que trabalhei junto.}         % Agradecimentos

% palavras-chave
\palchaves{Processos agéis, Desenvolvimento web, Frameworks.}
\resumo{Este relatório de estágio descreve as atividades desenvolvidas no Laboratório de Projetos e Estudos em Manejo Florestal – LEMAF. O estágio descrito neste relatório teve duração de 1 ano e teve como objetivo o desenvolvimento de sistemas e ferramentas para controle e manejo ambiental.
O relatório visa descrever os processos de desenvolvimento dentro da empresa, como metodologias ágeis, estágios de desenvolvimento (prototipação, design system, desenvolvimento, homologação) e organização no geral. Desta forma, o trabalho tem como objetivo relatar a experiência como desenvolvedor estagiário no LEMAF.}
% keywords devem vir antes do abstract
\keywords{Agile Processes, Web Development, Frameworks.} % keywords
\abstract{This internship report describes how activities are carried out at the Laboratory of Forest Management Studies and Projects - LEMAF. The internship described in this report lasted one year and aimed to develop systems and tools for environmental control and management. 
The report aims to describe the development processes within the company, such as methodologies, development steps (prototyping, system design, development, approval) and overall organization. Thus, the work aims to report an experience as a trainee developer at LEMAF.}% ################################ ##################
% Dados do guia
%\begin{titlepage}
%\pagestyle{empty}
%\renewcommand{\baselinestretch}{1}
%\enlargethispage{1.5cm}
%\input{reitoria}
%\cleardoublepage
%\end{titlepage}

%##################################################

% descomente para habilitar a lista desejada
\listoffigures                             % Lista de Figuras
%\listofilustracoes
%\listofgraficos							   % Lista de Gráficos
%\listoftables                              % Lista de Tabelas
%\listofquadros							   % Lista de Quadros
%\listofexemplos
%\listofteoremas
\tableofcontents                           % Sumário

\clearpage

\pagestyle{ufla}

%==============================================================================
% incluindo os capitulos
\chapter{INTRODUÇÃO}
\label{cap:introducao}

Este trabalho visa apresentar as experiências vividas durante o período como estagiário do LEMAF (Laboratório de Projetos e Estudos
em Manejo Florestal), laboratório situado dentro da Universidade Federal de Lavras (UFLA) onde são desenvolvidas soluções tecnológicas relacionadas a manejo florestal.
O laboratório foi fundado em 2004 e, desde lá, desenvolve diversos projetos para órgãos governamentais e empresas privadas, contando em 2018 com mais de 160 funcionários.

Os projetos do LEMAF geralmente possuem contextos relacionados à preservação ambiental, como monitoramento de áreas desmatadas e uso indevido de recursos hídricos. Porém sua carteira de projetos inclui diversos temas, como portais para povos indígenas, monitoramento de número de animais atropelados em determinada região e até mesmo gerenciadores de conteúdos.   
O estágio proporcionado pelo mesmo tinha como responsabilidades o seguimento de metodologias ágeis, organização, trabalho em equipe e, principalmente, o desenvolvimento de projetos web.

Por conta de uma grande rotação de projetos e times, foi necessário o estudo de diversos \textit{frameworks} e tecnologias, onde os principais relacionados com \textit{frontend} foram Angular, Vuejs e ReactJS, enquanto com \textit{backend} foram SpringBoot, DotNet \textit{Framework} e Play\textit{Framework}.
Para que o desenvolvimento dos projetos fluíssem efetivamente, eram utilizadas diversas metodologias ágeis, como principais \textit{Scrum} e \textit{Kanban}.

Com toda essa carga de conhecimento e responsabilidades, ser estagiário no LEMAF se tornou uma experiência incrível. Estava sempre evoluindo, conhecendo novas tecnologias do mercado e tendo oportunidade de obter conhecimento de pessoas extremamente experientes no ramo.

Os demais capítulos deste relatório estão organizados da seguinte forma.  No Capítulo 2 são abordadas as ferramentas e processos utilizados durante o estágio. No Capítulo 3 são apresentadas as atividades desenvolvidas durante o estágio. No Capítulo 4 são apresentadas conclusões.
\chapter{METODOLOGIAS E FERRAMENTAS}
\label{cap:elementos}

Para o desenvolvimento dos projetos foram utilizadas metodologias ágeis de desenvolvimento e ferramentas para desenvolvimento web (\textit{backend} e \textit{frontend}). Para entender melhor essas tecnologias as seções seguintes abordam cada uma delas.
\section{Metodologias Ágeis}

O "Manifesto ágil" surgiu em 2001 quando 17 conhecedores de metodologias ágeis se reuniram e discutiram como deveria ser o processo de desenvolvimento de software com metodologias ágeis, elencando 12 princípios que priorizam principalmente a satisfação do cliente, o trabalho em equipe, entregas de software mais rápidas e colaboração com o cliente. - \cite{manifestoAgil}

As metodologias ágeis são estratégias e comportamentos que devem ser seguidos afim de obter um resultado otimizado e mensurável. 
As principais metodologias ou \textit{frameworks} ágeis existentes são: 
Scaled Agile \textit{Framework} (SAFe) \cite{hayes2016scaling};
Feature Driven-Development (FDD)\cite{fdd};
Test Driven Development (TDD)\cite{beck2003test};
eXtreme Programming (XP) \cite{wildt2015extreme};
Dynamic Systems Development Method (DSDM) \cite{stapleton1997dsdm};
\textit{Kanban} \cite{boeg2010kanban};
entre outros.

No contexto do estágio descrito, \textit{Scrum} e \textit{Kanban} foram utilizadas e são descritas nas seções a seguir.
\subsection{\textit{Scrum}}

\textit{Scrum} é uma metodologia ágil que visa ter escopo fechado (número de tarefas predefinidas), calculadas através de métricas da equipe, como pontos de esforço.

Foi desenvolvida baseada nas metodologias de empresas do japão, como Toyota e Honda. É um \textit{framework} de trabalho que pode ser adaptado para diversas áreas.

As tarefas a serem desenvolvidas ficam armazenadas no \textit{product backlog}\footnote{Lista priorizada contendo as tarefas com uma breve descrição.}, sendo retiradas aquelas com maior prioridade para serem inseridas na \textit{Sprint}.

Também existem alguns ritos que devem ser seguidos para o bom funcionamento do \textit{Scrum}, como as \textit{Daily Meetings, Planning, Review Meeting e Retrospective}, que ocorrem em um fluxo como ilustrado na Figura \ref{fig:scrum}.

\begin{figure}[H]
\centering
\caption{Fluxo \textit{Scrum}} %legenda
\includegraphics[scale=0.15]{Scrum}\\  % o 0.9 indica 90% do tamanho original
% pdfLaTeX aceita figuras no formato PNG, JPG ou PDF
% figuras vetoriais podem ser exportadas para eps e depois convertidas para pdf usando epstopdf
{\small Fonte:https://blog.mjv.com.br/frameworks-ágeis-saiba-como-funcionam-na-prática} %Fonte da imagem
\label{fig:scrum} %rotulo para refencia
\end{figure}

\textit{Daily Meetings} são reuniões diárias, rápidas e práticas para atualização da equipe sobre em que cada um está trabalhando, para um melhor entendimento da equipe.

\textit{Review Meetings} são reuniões feitas após o fim de cada \textit{Sprint}, para entender os resultados da equipe durante a \textit{Sprint}.

\textit{Retrospective} é uma reunião que ocorre todo fim de \textit{Sprint} para identificar o que funcionou e o que pode ser melhorado na \textit{Sprint}.

\textit{Planning} é a reunião que deve ser feita antes de cada \textit{Sprint} para definir quais atividades devem ser desenvolvidas, entender melhor sobre elas, descrevendo-as minuciosamente e prevendo possíveis problemas.
Quem gerencia esse \textit{product backlog} é o  ~\nameref{sec:po}, que cuida de organizar, definir e priorizar as tarefas.

Muitas vezes separadas em \textit{Sprints} (tempo pré-definido pela equipe, geralmente de 10 dias úteis), as tarefas são incluídas em um quadro e devem ser finalizadas no prazo previsto.

\begin{quote}
  \textit{Scrum} é um \textit{framework} simples e pequeno e, assim, funciona bem em  cada contexto se for utilizado em conjunto com outras técnicas e praticas a serem experimentadas e adaptadas. - \cite{sabbagh2014scrum}
\end{quote}
\begin{quote}
  O \textit{Sprint} é o ciclo de desenvolvimento, onde o incremento do produto pronto é gerado pelo Time de Desenvolvimento a partir dos itens mais importantes do Product Backlog. - \cite{sabbagh2014scrum}
\end{quote}


\subsubsection{\textit{Scrum Master}}

Cargo que tem como função cuidar das obrigações impostas sobre a metodologia \textit{Scrum}, como lembrar dos ritos, marcar reuniões e garantir o bom desenvolvimento das atividades estabelecidas para a \textit{Sprint}.

\begin{quote}
Garante que o time esteja
totalmente funcional e
produtivo.

Facilita a colaboração entre as
funções e áreas e elimina os
impedimentos do time.

Protege o time de
interferências externas.

Garante que o processo está
sendo seguindo. Participando
das reuniões diárias, revisão
da \textit{Sprint}, e planejamento. \cite{sabbagh2014scrum}

\end{quote}

\subsubsection{Analista de Qualidade(Tester)}

Responsável por encontrar problemas e possíveis melhorias durante o desenvolvimento, garantido o funcionamento total do sistema para que seja validado com o cliente.

\subsubsection{Gerente de projetos(GP)}

Cargo que tem como função gerenciar os projetos e times da sua
 tribo\footnote{Conjunto de desenvolvedores e outros cargos que cuidam de um ou vários projetos.}, cuidando para que sejam entregues os requisitos no prazo combinado.

\begin{quote}
  O gerente de projetos é a pessoa destacada e designada como principal responsável por atingir os objetivos do projeto. - \cite{cruz2013scrum}
\end{quote}
\subsubsection{\textit{Product Owner(PO)}}
\label{sec:po}
Cargo que tem como função cuidar do relacionamento do time com o produto,
 definindo e priorizando requisitos.
 
 \begin{quote}
  Define os requisitos do
  produto, decide a data de
  release e o que deve conter
  nela.

  É responsável pelo retorno
  financeiro (ROI) do produto.
  
  Prioriza os requisitos de
  acordo com o seu valor de
  mercado.
  
  Pode mudar os requisitos e
  prioridades a cada \textit{Sprint}.

  o Aceita ou rejeita o resultado de
  cada \textit{Sprint}. - \cite{sabbagh2014scrum}
 \end{quote}

\subsection{\textit{Kanban}}
\textit{Kanban} é uma opção de metodologia ágil mais adaptativa, tendo escopo aberto torna-se possível inserir atividades durante o tempo.
Muitas vezes é comparado a uma tubulação de água, onde existe uma determinada quantidade de água a ser transportada, porém é necessário definir o tamanho do tubo ou vazão, sendo essa, a quantidade de esforço que a equipe consegue trabalhar.
Assim como o \textit{Scrum}, possui um backlog controlado por um Product Owner. As atividades são inseridas assim que libera a vazão, como em uma tubulação.
Diferentemente do \textit{Scrum}, não há necessidade de \textit{Reviews} e \textit{Retrospective} uma vez que não possui \textit{Sprints}, porém algumas reuniões para analisar o desempenho da equipe são um boa prática.
Geralmente é determinado um máximo de esforço e, ao final de uma atividade, é inserida uma nova com prioridade.

O \textit{Kanban}, como ilustra a Figura \ref{fig:kanban}, é um quadro onde as tarefas são organizadas em filas de acordo com seu estágio de execução: Novo, Pronto, Em Andamento, Pronto para testar, Em teste, Terminado.
Cada  tarefa é definida com um identificador, descrição da atividade, pessoa responsável e prioridade. As tarefas podem ser movimentadas de uma fila para outra de acordo com seu estado de execução.

\begin{figure}[H]
\centering
\caption{Quadro \textit{Kanban}} %legenda
\includegraphics[scale=0.2]{quadroKanban}\\  % o 0.9 indica 90% do tamanho original
% pdfLaTeX aceita figuras no formato PNG, JPG ou PDF
% figuras vetoriais podem ser exportadas para eps e depois convertidas para pdf usando epstopdf
{\small Fonte: https://taiga.ti.LEMAF.ufla.br/} %Fonte da imagem
\label{fig:kanban} %rotulo para refencia
\end{figure}

\section{\textit{Frameworks}}

Para facilitar o desenvolvimento, diversas ferramentas foram criadas e durante o desenvolvimento de novos projetos foi necessária a utilização das mesmas.

As aplicações que utilizam esses \textit{frameworks} geralmente são separadas em \textit{frontend} e \textit{backend} devido à arquitetura cliente-servidor que é utilizada para o desenvolvimento das aplicações. O cliente é o agente responsável por enviar mensagens requisitando algum serviço ao servidor e é conhecido como front-end, pois é a parte que interage diretamente com o usuário. 

O servidor é o agente responsável por realizar as operações necessárias para
responder as requisições do lado cliente da aplicação, sendo também conhecido
como \textit{backend}.
Essa definição facilita o desenvolvimento modularizado da aplicação, uma vez que o \textit{backend} pode ser utilizado por diversos \textit{frontend}.

\subsection{Frontend}

Nessa seção são apresentados frameworks para front-end utilizados pelo LEMAF.

\subsubsection{AngularJS}

AngularJS é um \textit{framework} open-source, de desenvolvimento front-end, que possibilita o desenvolvimento de aplicações web.

Teve sua primeira versão lançada em 2010 pela Google.

\begin{quote}
  o AngularJS foi criado por Miško Hevery e Adam Abron  em  2009  e  é  um  \textit{framework}  JavaScript open  source(código  aberto), client-side(do lado  do  cliente)  que  promove  uma  alta  produtividade  na  experiência  do  desenvolvimento Web - \cite{ferreira2018analise}
\end{quote}


\subsubsection{VueJS}

VueJS é um \textit{framework} JavaScript de open-source, focado no desenvolvimento de interfaces de usuário e aplicativos de página única.

Teve sua primeira versão lançada em 2014.

\begin{quote}
  Vue é uma lib/\textit{framework} JavaScript reativo, para  o  desenvolvimento  de  componentes  que,  por  sua  vez,  são  códigos  que  podem  ser reaproveitados em sua aplicação - \cite{ferreira2018analise}
\end{quote}


\subsubsection{ReactJS}

O React é uma biblioteca JavaScript, open-source, com foco em criar interfaces de usuário em páginas web. É mantido por empresas como Facebook e Instagram e uma comunidade de desenvolvedores individuais. 

Teve sua primeira versão lançada em 2013.

\begin{quote}
  Segundo a sua documentação(2018), ele é, na  verdade,  uma  biblioteca  de  UI  (User  Interface),  representando  apenas  a  camada view do Model  View  Controler(MVC).  - \cite{ferreira2018analise}
\end{quote}

\subsection{Backend}

É a parte da aplicação que é inacessível ao usuário, onde é controlado todo o sistema (autenticação, regras de negócio, jobs e etc).

\subsubsection{\textit{Play Framework}}

O Play \textit{Framework} é uma alternativa "limpa" de esticar as stacks do Java Enterprise. Ele se concentra na produtividade do desenvolvedor e tem como objetivo arquiteturas RESTful. 

O objetivo da estrutura do Play é facilitar o desenvolvimento de aplicativos da web, mantendo o Java.

Teve sua primeira versão lançada em 2009.

\begin{quote}
\textit{Play! Framework} 2 é planejado para ser "full stack" e completamente integrada, a boa notícia é que não há requisitos específicos para você ou meu ambiente começar a criar novos aplicativos da web. - \cite{petrella2013learning}
\end{quote}

\subsubsection{\textit{Spring Boot}}

O Spring é um \textit{framework} open-source para a plataforma Java criado por Rod Johnson e descrito em seu livro "Expert One-on-One: JEE Design e Development".
Trata-se de um \textit{framework} baseado nos padrões de projeto inversão de controle e injeção de dependência.

Teve sua primeira versão em 2002.

\begin{quote}
  Spring \textit{Framework} é um \textit{framework} voltado para desenvolvimento de aplicações corporativas para a plataforma Java, baseado nos conceitos de inversão de controle e injeção de dependências. - \cite{weissmann2014vire}
\end{quote}

\subsubsection{\textit{DotNet Framework}}

O \textit{.NET Framework} é uma iniciativa da empresa Microsoft, que visa uma plataforma única para desenvolvimento e execução de sistemas e aplicações.
Todo e qualquer código gerado para .NET pode ser executado em qualquer dispositivo que possua um \textit{framework} de tal plataforma.

Teve sua primeira versão lançada em 2002.

\begin{quote}
  
O .Net \textit{Framework} é um \textit{framework} de desenvolvimento que fornece uma nova interface de programação para serviços e APIs do Windows e integra várias tecnologias que surgiram da Microsoft no final dos anos 90. \cite{thai2003net}
\end{quote}

\subsection{Banco de Dados}

É aonde ficam armazenadas informações necessárias para o funcionamento das aplicações.

\subsubsection{Postgresql}

É um sistema gerenciador de banco de dados desenvolvido na linguagem de programação C.
Gerencia banco de dados relacionais e possui ótimo desempenho.

Possui extensões que contribuem com sua eficacia, como por exemplo o POSTGIS, uma extensão que possibilita o Postgresql a utilizar dados Geoespaciais.
Um das vantagens é que sua licença é gratuita desde fins estudantis a empresariais.
Teve sua primeira versão lançada em 2009.

\begin{quote}
  O PostgreSQL é uma das opções de banco de dados, pois se trata de um servidor SGBD de grande potencial e confiabilidade, contendo todas as características dos principais bancos de dados utilizados no mercado. Uma das suas características são suas licenças para uso gratuito, seja para fins estudantis seja para a realização de negócios, possibilitando que empresas o utilizem livremente. \cite{postgres}
\end{quote}

\section{TECNOLOGIAS}
\label{cap:conclusao}

Durante o estágio, tive a possibilidade de testar e utilizar diversas tecnologias e \textit{frameworks}, com foco especial em tecnologias \textit{frontend}.
Para melhor entendimento do trabalho, é necessário um dissertação sobre as mesmas.

Os \textit{Frameworks} principais de \textit{frontend} são ReactJS, Angular e Vuejs.
Sendo ReactJS e VueJS os principais em favoritos dentro da comunidade do GitHub\footnote{Comunidade de desenvolvedores onde se é possível compartilhar e discutir sobre projetos e tecnologias.} como demonstrado na figura \ref{fig:github}

\begin{figure}[H]
\centering
\caption{Comparação frameworks front-end pelo github} %legenda
\includegraphics[scale=0.55]{githubFramework}\\  % o 0.9 indica 90% do tamanho original
\label{fig:github} %rotulo para refencia
{\small Fonte: https://star-history.t9t.io} %Fonte da imagem
\end{figure}

Com seu lançamento tardio em relação ao ReactJS, o VueJS vem crescendo e neste ano de 2019 ultrapassou o numero de estrelas no github, enquanto o \textit{framework} Angular
 cresce de forma lenta e contínua.

Os \textit{frameworks} VueJS e ReactJS tinham como grande diferencial o uso da Virtual DOM\footnote{É um framework para manipulação do DOM} em vez da DOM(\textit{Document Object Model})\footnote{é a representação dos compomentes na página}. Uma vez que a DOM era uma abstração do codigo HTML, a DOM Virtual é uma abstração da DOM.

Enquanto a DOM é uma árvore de objetos interligados, com diversas conexões e ramificações, onde modificações simples nas mesmas muito frequentes, a DOM virtual já possui configurações e implementações do \textit{browser}, livrando o desenvolvedor de ter que configurá-las manualmente. Além disto a DOM virtual é mais leve e simples, sendo mais fácil compreendê-la e desenvolvê-la.

Outro grande diferencial graças a esta mudança, é que a abstração de componentes se tornou mais inteligente, uma vez que era possível desenvolver componentes desacoplados e acoplados somente onde eram necessários.

Em diversos projetos que tive contato, era utilizado Angular, por conta de ter sido lançado em 2010 pela Google. Foi uma grande evolução, uma vez que projetos mais velhos foram desenvolvidos em Flex, que era uma tecnologia para interfaces em flash com integração com Java.

O \textit{framework} Angular facilitava absurdamente o trabalho como desenvolvedor, uma vez que em diversos casos, que seriam necessárias diversas linhas de Jquery e Javascript, era somente necessário o uso de uso de uma, como exemplificam os códigos da Figura \ref{fig:exemplocodigo1} e a Figura \ref{fig:exemplocodigo2} que adicionam um evento a elementos DOM pelo JQuery e VirtualDOM pelo Angular.


\begin{figure}[!htb]
\centering
\caption{Adicionando evento a elementos DOM pelo JQuery} %legenda
\begin{lstlisting}
var box = $( "#box" );
$( "#botao" ).on( "click", function( event ) {
    box.show();
});
\end{lstlisting} 
\label{fig:exemplocodigo1} %rotulo para refencia
\end{figure}

\begin{figure}[!htb]
\centering
\caption{Adicionando evento a elementos VirtualDOM pelo Angular} %legenda
\begin{lstlisting}
<div ng-show={show} ng-click={() => {show = true}}/>
\end{lstlisting} 
\label{fig:exemplocodigo2} %rotulo para refencia
\end{figure}
    

Como a base de desenvolvedores no LEMAF já possuia muita experiência com Angular, foi difícil fazer com que adotassem outras tecnologias em outros projetos.

Porém logo após iniciativa e insistência de um desenvolvedor, apresentamos o VueJS para a organização do LEMAF e conseguimos que os projetos fossem desenvolvidos com ele.

O VueJS não possuia documentação vasta como o Angular, porém possuia atualizações recorrentes, porém com sua ascensão, foi fácil demonstrar que em alguns anos a documentação dele seria melhor que a do Angular.

Porém era complicado mesmo assim, pois a credibilidade de um \textit{framework} da Google contra a de um ex-desenvolvedor da Google(Criador do VueJS) era incomparável, era realmente um risco a mudança. 
Mas ao mostrar que o projeto recebia uma grande verba de patrocínio, a curva de crescimento e que o mesmo possuia uma equipe fixa de pelo menos 50 contribuidores, entenderam que o risco não era tão grande.

A adesão ao VueJS foi muito bem aceita em diversas equipes, pois a flexibilidade que ele proporciona ao desenvolvedor comparado ao Angular era gigantesca, com a possibilidade de configuração quase infinita, diversas bibliotecas e varios modos de arquitetura, o VueJS se tornou foco de estudo na empresa.
Uma das iniciativas vindas dos organizadores da empresa foi fornecer alguns cursos da plataforma Alura para desenvolvimento web com VueJS.

Seu tamanho mesmo com sendo maior que o ReactJS, demonstrava mais fluidez, desde a ferramenta de \textit{hot-reload}\footnote{Recarregamento quente, possibilita que a DOM seja recarregada somente nos locais de modificação, sem ser necessário o recarregamento da tela total.}, até a compilação para modo produção.
Sua arquitetura era simples e direta, centralizada em poucos arquivos e ao mesmo tempo separava bem os comportamentos. Sua documentação era bem escrita, descrevendo os ciclos de vida dos componentes até suas diretivas.

Muitos aprenderam por conta própria facilmente, uma vez que a complexidade abaixava, tendo toda a estrutura de um componente do Angular com 5 arquivos, em 1 arquivo .vue somente.
Como vários módulos do LEMAF já utilizavam de Pug/Jade em vez do HTML puro, a transição foi muito tranquila, pois VueJS já suportava Pug e diversas outras linguagens.

Outro grande benefício encontrado durante a transição foram as bibliotecas de UI\footnote{User Interface Design - é uma área de design que planeja e analisa os modos de iteração do usuario com o sistema.}, pois com Angular o mais adequado era o uso do Material Design que já era da própria empresa ou 
Bootstrap\footnote{Framework para desenvolvimento de componentes e front-end para sites e aplicações web usando HTML, CSS e JavaScript.}, porém com o VueJS vieram varias outras bibliotecas como Vuetify e ElementUI, que tornaram as interfaces bem mais leves, modernas e bonitas, saindo daquele estilo antigo e estático.

Também foi apresentado pela equipe de design, eu e outro desenvolvedor, a ideia de Design System para a organização, que era algo que seria fácil de ser aplicado graças ao VueJS, que tinha um ótimo suporte a componentização e que se usado 
de forma correta(Separando componentes de visualização e lógica), poderia melhorar significantemente o tempo para desenvolvimento dos projetos.

Em 2019, vendo o advento das tecnologias de desenvolvimento hibrido para mobile, foi questionado a equipe se seria interessante o uso de \textit{frameworks} mais próximos, para possíveis futuros projetos.
Daí a adesão ao ReactJS, já que o principal utilizado para desenvolvimento mobile era o ReactJS-native, que se assemelhava bastante.

Ao estudarmos sobre, entendemos que a curva de aprendizado para quem desenvolvia VueJS iria ser bem pequena, já que ambos eram bem flexíveis.

Ele era mais leve que o VueJS que já era muito leve e ainda possuia uma documentação muito boa.
porém um grande diferencial que existia era o uso do JSX, uma versão do Javascript que combina elementos do HTML dentro do Javascript. Era bem legível e bem similar ao HTML, mas não tinha um bom suporte para Pug.

Mesmo com este problema, resolvemos testar um projeto pequeno com ele. 

As primeiras semanas foram difíceis, tivemos que estudar muito sobre o \textit{framework} pois ele possuia diversas bibliotecas que melhoravam significantemente meu uso.

Após as duas primeiras semanas o desenvolvimento fluiu, as duvidas eram pouco recorrentes e as respostas eram fáceis de encontrar. porém voltamos ao problema inicial do Angular, 
a falta de um biblioteca de UI que combinasse com a identidade da empresa.

Foi colocado então em prática a construção do Design System, um processo de construção de identidade visual para empresa, com componentes próprios desenvolvidos com excelência para poderem ser utilizados em varios projetos, facilitando o desenvolvimento de todos e garantindo qualidade.

A criação dessa biblioteca tinha como objetivo fazer com que os desenvolvedores centralizassem os componentes, contribuindo publicamente com a biblioteca, sendo mais fácil corrigir um erro e depois atualizar a biblioteca que resolve-lo em todos projetos.

E isto tornou-se prazeroso com ReactJS, pois o suporte para componentização por ele é ótimo, desde o desenvolvimento até a criação dos testes automatizados.

\chapter{Atividades desenvolvidas}
\label{cap:desenvolvimento}

Nos primeiros dias como estagiário no LEMAF, foi necessário realizar treinamentos online para que entendesse como funcionava as principais tecnologias utilizadas lá.
Forneceram uma licença na plataforma Alura para que aprendesse a utilizar o \textit{framework} VueJS e Spring e diversos livros sobre os processos \textit{Scrum} e banco de dados.

Um tutor foi designado para criar um quadro de tarefas que deveriam ser cumpridas para que tivesse o conhecimento básico para desenvolver os projetos.
Após terminar as tarefas, foi passado um projeto para que fosse possível praticar e verificar os conhecimentos obtidos.

Na sua grande maioria, os projetos do LEMAF são sistemas web\footnote{Sistemas web são soluções de software que podem ser acessados através de um navegador  em uma rede de computadores interna de uma empresa ou pela Internet.
}, constituídos de uma aplicação \textit{backend} e uma \textit{frontend}.

Durante o desenvolvimento, o \textit{backend} é, geralmente, compilado no computador do próprio desenvolvedor, porém quando há necessidade de serem feitos testes, é criado um servidor dentro da rede local para que possam ser efetuados os mesmos, sendo mais precisos por conta de estarem em um servidor.

A infraestrutura de servidores\footnote{Computadores físicos ou virtuais que executam e servem a aplicação} do LEMAF são servidores com diversas máquinas, todas rodando NGNIX\footnote{Disponivel em: https://www.nginx.com/} para prover as aplicações.

O \textit{frontend} seguia o mesmo processo do \textit{backend}, porém quando era enviado para alguma máquina externa, era criada alguma configuração para que o \textit{backend} servisse o \textit{frontend}, evitando assim problemas de CORS\footnote{CORS  (Cross-Origin Resource Sharing), é uma regra criada para segurança dos usuários ao  usar navegadores (\url{https://developer.mozilla.org/pt-BR/docs/Web/HTTP/Controle_Acesso_CORS})}.

Já o banco de dados\footnote{Estrutura de armazenamento de dados} era controlado por um DBA\footnote{Database Administrator, gerencia a estrutura do banco de dados, desde a escolha do gerenciador de banco de dados até a estrutura interna do mesmo.}, que criava e gerenciava toda a estrutura de banco, sendo somente necessário aos desenvolvedores discutir melhores soluções e criar demandas para os mesmos realizarem e contemplarem suas necessidades.

Para a maioria dos bancos era utilizada o SGBD\footnote{ Sistemas de Gestão de Base de Dados} Postgres\footnote{Disponivel em: https://www.postgresql.org/}, por conta de sua extensão POSTGIS que dava suporte a diversos tipos de dados relacionados a geometrias.

Como o foco do LEMAF é meio ambiente, o uso de mapas era recorrente na empresa, sendo necessário então obter um conhecimento básico sobre termos ambientais como APP e AUR.

\begin{quote}
    Área de preservação permanente - "área protegida, coberta ou não por vegetação nativa, com a função ambiental de preservar os recursos hídricos, a paisagem, a estabilidade geológica e a biodiversidade, facilitar o fluxo gênico de fauna e flora, proteger o solo e assegurar o bem-estar das populações humanas"\cite{brasil2012}
\end{quote}

\begin{quote}
    Áreas de Uso Restrito -  área localizada no interior de uma propriedade ou posse rural, delimitada nos termos do art. 12, com a função de assegurar o uso econômico de modo sustentável dos recursos naturais do imóvel rural, auxiliar a conservação e a reabilitação dos processos ecológicos e promover a conservação da biodiversidade, bem como o abrigo e a proteção de fauna silvestre e da flora nativa;
\cite{brasil2012}

\end{quote}

Durante o período de estagiário participou de todos squads, trabalhando com diversos times\footnote{Grupo de funcionários, geralmente formado por desenvolvedores, analistas de qualidade e Product Owners}, projetos e tecnologias.
E com isso, foi possível obter experiências com diversas tecnologias diferentes.
\begin{figure}[H]
\centering
\caption{Fluxo de aprendizado} %legenda
\includegraphics[scale=0.35]{fluxoAprendizado}\\  % o 0.9 indica 90% do tamanho original
% pdfLaTeX aceita figuras no formato PNG, JPG ou PDF
% figuras vetoriais podem ser exportadas para eps e depois convertidas para pdf usando epstopdf
\label{fig:exemplo} %rotulo para refencia
\end{figure}

\section{SICAR}

Em sua primeira equipe (Squad 1 - Carreta Furacão), tive como tecnologias necessárias JAVA (\textit{backend}) e Angular (Frontend) para evoluir os projetos do Cadastro Ambiental Rural, incluindo os projetos SICAR, Central do Responsável Técnico, Central do Proprietário, PRA-OFF.

Meu trabalho nesta equipe era desenvolver novas funcionalidades no sistema, como inserir novos dados no arquivo .PRA gerado pelo PRA-OFF, criptografa-lo e descriptografa-lo nas centrais.

Foi onde também tive meu primeiro contato com o Banco de Dados, uma vez que ainda não havia feito a disciplina deste assunto e então não possuia domínio ou conhecimento necessário, porém sua equipe possuia desenvolvedores extremamente experientes e muito dispostos a ensinar.
tive diversas atividades onde era necessário a criação de comandos SQL e como todos passavam por validação do DBA, tinha recorrentemente que corrigir os mesmos, entendendo e aprendendo com meus erros.

Também tive o obstaculo de nunca ter utilizado alguma IDE, vinha programando em editores de texto desde o começo da graduação, então ao ter a necessidade de programar com eficiência, adotei o uso da IDE Eclipse, que já era utilizada pela equipe para desenvolvimento \textit{backend}.

O \textit{backend} era uma parte que conseguia entender, uma vez que já tinha experiências com JAVA graças a disciplina de Programação Orientada a Objetos e desenvolvimento Mobile, onde tive que desenvolver nativo para Android utilizando JAVA.

Para desenvolver e reproduzir o fluxo do sistema, foi necessário aprendizado sobre tecnologias relacionadas a georeferenciamento e mapas, como ARCGIS\footnote{Disponivel em: \url{https://www.arcgis.com/index.html}}, leaflet\footnote{Disponivel em: \url{https://leafletjs.com/}} e GEOSERVER\footnote{Disponivel em: \url{http://geoserver.org/}}.

ARCGIS era uma ferramenta para desenho e exibição de geometrias, sendo possível reproduzir desde fazendas pequenas(menos de 4 módulos fiscais\footnote{II - Pequena Propriedade - o imóvel rural:

a) de área compreendida entre 1 (um) e 4 (quatro) módulos fiscais;\cite{brasil1993}}) a áreas enormes.

A leaflet era a principal biblioteca utilizada para mapas dentro do LEMAF, uma vez que era grátis e eficiente, com ampla documentação e compatibilidade. A ferramenta era de extrema importância para os projetos.

E para que ficasse centralizado diversos recursos de mapa, como as geometrias de APP do estado do PArá, hidrografia e outros, era utilizado o GEOSERVER, uma ferramenta que armazenava, gerenciava e provia geometrias.

\begin{figure}[H]
\centering
\caption{SICAR-PA} %legenda
\includegraphics[scale=0.3]{SICAR}\\  % o 0.9 indica 90% do tamanho original
% pdfLaTeX aceita figuras no formato PNG, JPG ou PDF
% figuras vetoriais podem ser exportadas para eps e depois convertidas para pdf usando epstopdf
{\small Fonte: \url{http://car.semas.pa.gov.br/}} %Fonte da imagem
\label{fig:sicar} %rotulo para refencia
\end{figure}

O projeto do SICAR(Figura \ref{fig:sicar})\footnote{Sistema de Cadastro Ambiental Rural, uma plataforma relacionada ao governo e que gerenciava os cadastros.} tinha como objetivo informar e controlar os cadastros ambientais rurais feitos no estado do Pará, sendo necessárias algumas integrações com os módulos do SICAR federal e as Centrais ligadas ao PRA.

\begin{quote}
    Para adequação dos imóveis rurais à nova legislação, foi criado o Cadastramento
Ambiental Rural (CAR) como registro público eletrônico de âmbito nacional,
obrigatório para todos os imóveis rurais, com a finalidade de integrar as informações ambientais das propriedades e posses rurais, compondo base de dados
para controle, monitoramento, planejamento ambiental e econômico e combate
ao desmatamento. - \cite{de2015cadastro}
\end{quote}

\section{PRA}
\begin{figure}[H]
\centering
\caption{PRA-OFF} %legenda
\includegraphics[scale=0.5]{pra-off}\\  % o 0.9 indica 90% do tamanho original
% pdfLaTeX aceita figuras no formato PNG, JPG ou PDF
% figuras vetoriais podem ser exportadas para eps e depois convertidas para pdf usando epstopdf
{\small Fonte: \url{http://www.cprh.pe.gov.br/Controle_Ambiental/Sistema%20Nacional%20de%20Cadastro%20Ambiental%20Rural%20-%20SICAR/PRA/43052%3B53356%3B480802%3B0%3B0.asp}} %Fonte da imagem
\label{fig:pra} %rotulo para refencia
\end{figure}

Já o projeto do PRA(Figura \ref{fig:pra})\footnote{Programa de regularização ambiental, responsável por regularizar situações de desmatamentos e outras inflações ambientais}, não eram possíveis tais integrações, pois ele era um módulo offline, sendo possível a utilização sem internet. Para o desenvolvimento do mesmo, foi necessário estudar sobre electron\footnote{Disponivel em: https://electronjs.org/}, uma ferramenta que possibilita criar módulos web em aplicações offline.

\section{Consulta Pública}
Após 5 meses a equipe foi dividida em duas e então foi feita uma redistribuição de projetos, quando fui alocado a tribo\footnote{Grupo de funcionários divido por equipes, gerenciado por um GP} Runners. Os projetos principais que contribui durante esse período foram o Consulta Pública - PARÁ e relatórios do PRA.

\begin{quote}
    O objetivo principal é promover as boas
    práticas de manejo florestal, por meio do enriquecimento das florestas secundárias
    remanescentes, localizadas fora das APPs, geralmente compondo a RL. - \cite{de2015cadastro}
\end{quote}
\begin{figure}[H]
\centering
\caption{Consulta publica} %legenda
\includegraphics[scale=0.22]{consulta-publica}\\  % o 0.9 indica 90% do tamanho original
% pdfLaTeX aceita figuras no formato PNG, JPG ou PDF
% figuras vetoriais podem ser exportadas para eps e depois convertidas para pdf usando epstopdf
{\small Fonte: \url{http://sistemas.semas.pa.gov.br/pra/consultaPublica/#/}} %Fonte da imagem
\label{fig:consultaPublica} %rotulo para refencia
\end{figure}

O projeto Consulta Pública(Figura \ref{fig:consultaPublica}) foi o primeiro projeto que tive como atividade a refatoração, pois se tratava de um projeto legado, utilizava uma das primeiras versões de VueJS no \textit{frontend} e tinha o layout bem ruim.
Foi a primeira obrigação que tive total responsabilidade, pois tinha que migrar todo o sistema para VueJS 2.0 e refatorar o \textit{frontend}.

Como era sua primeira experiência com o \textit{framework} VueJS, foi demandado mais tempo do que o usual, mas tive suporte do meu time que me auxiliava sobre arquitetura, padrões de projetos e boas práticas de programação.
Para melhor aprendizado e controle sobre códigos ruins, todas modificações feitas por mim eram submetidas ao Code Review\footnote{Revisão do código, consiste em algum ou alguns membros da equipe analisando meu código e fazendo comentários de melhorias} enviadas para o projeto apenas perante aprovação. 

O projeto Consulta Pública tinha como objetivo ilustrar aos proprietários de imóveis rurais suas áreas desmatadas, se seus imóveis estavam ou não de acordo com as regularizações ambientais e informações gerais sobre a geometria e hidrografia do terreno.

\begin{figure}[H]
\centering
\caption{Relatórios do PRA} %legenda
\includegraphics[scale=0.22]{relatorios-pra}\\  % o 0.9 indica 90% do tamanho original
% pdfLaTeX aceita figuras no formato PNG, JPG ou PDF
% figuras vetoriais podem ser exportadas para eps e depois convertidas para pdf usando epstopdf
{\small Fonte: \url{http://sistemas.semas.pa.gov.br/pra/relatorios/#/dados-gerais-adequacao-ambiental}} %Fonte da imagem
\label{fig:relatorios} %rotulo para refencia
\end{figure}

\section{Relatórios do PRA}
O projeto de relatórios(Figura \ref{fig:relatorios}) foi o primeiro projeto que tive participação desde o início, com uma equipe formada de 4 pessoas (2 desenvolvedores, uma tester e uma PO).
O projeto consistia em uma plataforma de relatórios sobre o Programa de Regularização Ambiental e Adequação Ambiental.

Foi o primeiro momento que tive que realmente aprender e utilizar dos conhecimentos relacionados a Banco de dados, e como havia feito a disciplina de Banco de Dados 1 e estava cursando Banco de Dados 2, foi de extrema importância os aprendizados das mesmas,
uma vez que lhe foi explicado melhor como funcionava os bancos e até mesmo como funcionava os bancos relacionados a geometrias espaciais.

O time tive autonomia de escolha da tecnologia, então pela ótima experiência com VueJS e o alto desempenho de tal, este foi o escolhido para o \textit{frontend}.
Porém, pelo baixo conhecimento sobre \textit{backend}, a escolha da tecnologia foi feita pelo outro desenvolvedor da equipe, que escolheu Spring Boot.
Como sua experiência se limitava apenas a evolução de software, houve diversos gargalos no desenvolvimento, como criação de ambientes, scripts para automatização de deploy entre outros.

Em meio a este projeto, descobriram um problema gigantesco no primeiro sistema que tive contato, o PRA-OFF. Ele quem empacotava e compactava as geometrias para o arquivo .PRA, porém, quando existiam muitas geometrias, o arquivo PRA podia ter tamanhos acima de 1GB.
O outro módulo, central do Responsável Técnico, que recebia os arquivos PRA, não aguentava processar o arquivo, travando no envio do mesmo.

Foi então sua responsabilidade desenvolver uma solução prática para o mesmo. Ná época, estava cursando a matéria de Programação Paralela e Concorrente, dai tive a ideia de otimizar este processo atravez de threads e processos, porém só isso não resolvia o problema.
Após liberar bastante poder de processamento para o servidor e criar diversas otimizações no processo de criação do PRA, foi possível diminuir o arquivo para aproximadamente 1/4 de meu tamanho original.

Isso se tornou possível através de consultas a pessoas mais experientes com dados geográficos, que me sugeriram converter as geometrias de GEOMETRY para KML.

Mas após 2 meses, por necessitarem de um desenvolvimento com maior domínio de \textit{frontend}, fui transferido para uma equipe especial, pois a tal não possuía tribo determinada.
O projeto era uma POC (Prova de conceito) direcionada a uma empresa do ramo de agronomia. Com prazos curtíssimos e complexidade alta, o projeto foi um dos mais difíceis que já trabalhei.
O projeto foi construído utilizando componentização, com o \textit{backend} feito com uma arquitetura bem definida e documentada para que fosse possível evoluir com o menor nível de dificuldade possível. Devido ao começo bem estruturado e documentado, o projeto foi bem sucedido.

Após esse projeto, fui alocado na tribo Atlântida, onde trabalhei juntamente com mais um desenvolvedor em no projeto SEIRH-CMS.

\section{SEIRH-CMS}

O projeto do SEIRH-CMS(Figura \ref{fig:seirhCms}) era um gerenciador de conteúdo da aplicação SEIRH\footnote{Sistema Estadual de Informações Sobre Recursos Hídricos do Pará, responsável sobre oferecer informações sobre os projetos e programas relacionados a recursos hídricos do pará}.
\begin{figure}[H]
\centering
\caption{SEIRH-CMS} %legenda
\includegraphics[scale=0.22]{SEIRH-CMS}\\  % o 0.9 indica 90% do tamanho original
% pdfLaTeX aceita figuras no formato PNG, JPG ou PDF
% figuras vetoriais podem ser exportadas para eps e depois convertidas para pdf usando epstopdf
{\small Fonte: \url{http://monitoramento.semas.pa.gov.br/SEIRHCMS}} %Fonte da imagem
\label{fig:seirhCms} %rotulo para refencia
\end{figure}
A plataforma web do SEIRH(Figura \ref{fig:seirh}) já existia, porém não havia comunicação com um \textit{backend}, então meu conteúdo não era gerenciável. Logo, havia a necessidade de refatoração total do sistema.

\begin{figure}[H]
\centering
\caption{SEIRH} %legenda
\includegraphics[scale=0.22]{SEIRH}\\  % o 0.9 indica 90% do tamanho original
% pdfLaTeX aceita figuras no formato PNG, JPG ou PDF
% figuras vetoriais podem ser exportadas para eps e depois convertidas para pdf usando epstopdf
{\small Fonte: \url{http://monitoramento.semas.pa.gov.br/SEIRH}} %Fonte da imagem
\label{fig:seirh} %rotulo para refencia
\end{figure}

Então, a refatoração do sistema foi atribuída como tarefa designada ao nosso time.
Como meu conhecimento sobre \textit{backend} já era mais amplo e devido a tribo Atlântida ser constituída de desenvolvedores DotNet, resolvemos utilizar o \textit{framework} DotNet Core 2.0, que 
provia diversas funcionalidades e atendia as nossas expectativas e necessidades.
Já o \textit{frontend} continuou sendo feito com VueJS, uma vez que nossas experiências com tal \textit{framework} eram recentes.

O prazo para entrega do projeto era curtíssimo de duas \textit{Sprints}(1 mês), então optamos por utilizar um quadro \textit{Kanban} para gerenciar as atividades.
Esta foi sua primeira experiência exercendo liderança em estabelecer as prioridades e definir as atividades, definição das atividades e organização no geral.

O projeto foi finalizado com excelência e no prazo estipulado, o que me rendeu nova oportunidade de alocação em um time que já trabalhava com DotNet e necessitava de mais um desenvolvedor.

O projeto desta vez fazia parte de um grande complexo de plataformas que havia sido encomendado por uma empresa de agronomia e tinha muitos componentes de \textit{frontend} similares entre sí.

Desse fato ocorreu uma iniciativa sua e de outro desenvolvedor de iniciarmos a construção do Design System em um contexto organizacional.

\section{Cria Design System}
Surgiu então, o Cria Design System(Figura \ref{fig:criaDesign})\footnote{Disponível em \url{https://criatecnologiainovacao.github.io/cria-design-ReactJS/}}, que é uma biblioteca de componentes UI\footnote{User Interface Design - é uma área de design que planeja e analisa os modos de iteração do usuario com o sistema.} para ReactJS.

Todos os componentes eram criados pelo design do projeto, definindo comportamentos e animações, desenvolvidos depois. Para a visualização individual dos componentes, utilizamos o StoryBook\footnote{Disponivel em: \url{https://storybook.js.org/}},
que cria e organiza o catálogo de componentes e então, para compartilhar tal biblioteca para toda a empresa e para que não surgissem bugs, foram implementados testes automatizados, 
uma vez que já havia tido contato com testes automatizados na disciplina de Engenharia de Software, e havia sido encorajado a desenvolver os mesmo em Desenvolvimento Web, resolvi que seria interessante a criação de testes de componentes, testando visualmente e comportamentalmente,
todos componentes e com necessidade mínima de cobertura de 80\%.

\begin{figure}[H]
\centering
\caption{Cria Design System} %legenda
\includegraphics[scale=0.3]{cria-design}\\  % o 0.9 indica 90% do tamanho original
% pdfLaTeX aceita figuras no formato PNG, JPG ou PDF
% figuras vetoriais podem ser exportadas para eps e depois convertidas para pdf usando epstopdf
{\small Fonte: \url{https://criatecnologiainovacao.github.io/cria-design-ReactJS/}} %Fonte da imagem
\label{fig:criaDesign} %rotulo para refencia
\end{figure}

Após auxiliar com o início do projeto e facilitar o desenvolvimento do mesmo, fui incorporado a uma equipe da mesma tribo, que cuidava de alguns projetos como o SIOUT e SIGERH-PA.

\section{SIGERH-PA}

O projeto do SIGERH-PA(Figura \ref{fig:sigerhpa}) (Sistema de Gerenciamento de Recursos Hídricos do Pará) é uma plataforma de cadastro e regularização de recursos hídricos (poços artesianos, nascentes e outros).
Nele usávamos como \textit{backend} o \textit{framework} DotNet \textit{Framework} 4.0 e como \textit{frontend} AngularJS.

\begin{figure}[H]
\centering
\caption{SIGERHPA} %legenda
\includegraphics[scale=0.222]{sigerpa}\\  % o 0.9 indica 90% do tamanho original
% pdfLaTeX aceita figuras no formato PNG, JPG ou PDF
% figuras vetoriais podem ser exportadas para eps e depois convertidas para pdf usando epstopdf
{\small Fonte: \url{http://sistemas.semas.pa.gov.br/sigerhpa/}} %Fonte da imagem
\label{fig:sigerhpa} %rotulo para refencia
\end{figure}

O sistema tinha complexidade gigantesca, uma vez que para cada tipo de recurso hídrico, era necessário um fluxo especifico de cadastro.

Uma das atividades mais importantes que fui responsável foi o desenvolvimento de um novo fluxo, totalmente diferente dos outros, então o reaproveitamento de código foi baixíssimo, somente pude reaproveitar alguns componentes.
Ao final do meu estágio, tive como tarefas principais: aprimorar e corrigir tal sistema, que tinha
vários problemas, se tratando de um sistema bastante grande.
\chapter{CONCLUSÃO}
\label{cap:conclusao}

Toda minha experiencia no Laboratório foi excepcional, obtive grande desenvolvimento profissional e pessoal, tendo que lidar com diversos problemas e pessoas.
Quando iniciei o estagio, meu conhecimento era somente relacionado ao que me foi ensinado na universidade ate o quinto período, oque na sua grande maioria eram conhecimentos sobre otimizações e códigos baixo nível.
Após alguns meses como estagiário, já possuía propriedade intelectual para opinar em decisões estruturais de projetos, tecnologias e ate mesmo dar prazos, logico que Após vários erros e muito estudo.
 
Os profissionais que tive o prazer de trabalhar sempre foram muito agradáveis e sempre agregaram muito no meu desenvolvimento.

No fim do meu estágio, posso dizer que já possuía propriedade para opinar em decisões importantes da empresa, como quais deviam ser as tecnologias principais e bibliotecas utilizadas.

Graças aos problemas resolvidos e projetos desenvolvidos no LEMAF, fui capaz de desenvolver minhas habilidades como desenvolvedor web e criar grande portfólio, abrindo diversas portas para o mercado de trabalho.
O estagio também me fez ter uma visão de todos os cargos e caminhos que poderia trilhar dentro da minha área de estudo, fazendo com que eu decidisse que realmente tinha como desejo e vocação ser desenvolvedor.

Meu atual emprego na empresa Equals só me foi oferecido por conta do conhecimento obtido em ReactJS e Spring. 

Posso dizer que os principais motivos de minha contratação foram minhas experiencias com as tecnologias e principalmente o desenvolvimento do Cria Design.

\begin{figure}[H]
\centering
\caption{Bilhete recebido da confraternização no fim de 2018} %legenda
\includegraphics[scale=0.3]{agradecimento}\\  % o 0.9 indica 90% do tamanho original
% pdfLaTeX aceita figuras no formato PNG, JPG ou PDF
% figuras vetoriais podem ser exportadas para eps e depois convertidas para pdf usando epstopdf
\label{fig:exemplo} %rotulo para refencia
\end{figure}

%==============================================================================
% Incluindo bibliografia
%\bibliographystyle{plain}             % estilo para labels em numeros
%\bibliographystyle{alpha}             % estilo para labels em iniciais
\bibliographystyle{abntex2-alf}           % estilo para referências usando ABNT, 
                                       % precisa instalar o abntex para usar!!!

%inclui Referências Bibliográficas
%inclui Referências Bibliográficas
\referencias
\bibliography{refbib}			% arquivo exemplo refbib.bib
%==============================================================================
% Incluindo anexos num1erados com letras maiusculas.
%\apendices
\include{apendice1}


%==============================================================================
% Fim do texto
\end{document}
